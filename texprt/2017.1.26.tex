%==========================================================================
%Template File
%   Copyright (C) 2006-2009                                
%           by Shinya Watanabe(sin@csse.muroran-it.ac.jp) 
%==========================================================================
\documentclass[a4j,titlepage]{jarticle}
\usepackage{programing_report}
\usepackage{itembkbx}
\begin{document}

%--------------------
%以下に実験レポートのタイトル,自分のクラス名,学籍番号,氏名,提出日を書く.
%--------------------

%%\title{レポートタイトル}を記述する.
\title{第12回 プログラミング演習 レポート}

%%\author{クラス名}{学籍番号}{氏名}を記述する.
\author{前半クラス16024003}{赤堀 冴太朗}

%%\date{提出する年月日}を記述する.
\date{2017 年 1 月 19 日}
\maketitle

%--------------------
%以下から本文を開始する.
%--------------------

\section{総合演習 lecture 8-1}
\subsection{ソースコード}
\begin{breakitembox}[l]{ソースコード}
\begin{verbatim}

/*****
name e81-2.c
do アドレス帳の作成
in char型配列構造体
out   〃
author 16024003
day 2016.12.7
other none
*****/
#include<stdio.h>
void test(void)

struct phone_book
{
    char name[31];
    char secname[31];
    char maile[31];
    char phone[14];
}a[10];

void test(void)
{
    int i;
    printf("アドレス帳を開きます。名前(半角30文字)メールアドレス(半角30文字)電話番号(半角13文字)を入力してください。\n");
    for(i=0;i<10;i++)
    {
        printf("%d人目 名前(半角30文字)を入力。姓と名はスペースで区切ってください。",i+1);
        scanf("%30s",a[i].name);
        scanf("%30s",a[i].secname);
        printf("%sのメールアドレス(半角30文字)を入力。",a[i].name);
        scanf("%30s",a[i].maile);
        printf("%sの電話番号(半角13文字)を入力。",a[i].name);
        scanf("%13s",a[i].phone);
    }
    for(i=0;i<10;i++)
    {
        printf("%s ",a[i].name);
        printf("%s ",a[i].secname);
        printf("%s ",a[i].maile);
        printf("%s\n",a[i].phone);

    }
}
\end{verbatim}
\end{breakitembox}

\subsection{実行結果}
\begin{itembox}[l]{実行結果}
\begin{verbatim}
これからmain内を作成

\end{verbatim}
\end{itembox}



\end{document}























