%==========================================================================
%Template File
%   Copyright (C) 2006-2009                                
%           by Shinya Watanabe(sin@csse.muroran-it.ac.jp) 
%==========================================================================
\documentclass[a4j,titlepage]{jarticle}
\usepackage{programing_report}

\begin{document}

%--------------------
%以下に実験レポートのタイトル,自分のクラス名,学籍番号,氏名,提出日を書く.
%--------------------

%%\title{レポートタイトル}を記述する.
\title{第7回 プログラミング演習 レポート}

%%\author{クラス名}{学籍番号}{氏名}を記述する.
\author{前半クラス16024003}{赤堀 冴太朗}

%%\date{提出する年月日}を記述する.
\date{2016 年 12 月1 日}
\maketitle

%--------------------
%以下から本文を開始する.
%--------------------

\section{基礎課題1 Lecture5演習1-3}
\subsection{ソースコード}
\begin{itembox}[l]{ソースコード}
\begin{verbatim}
/*****
name e111_2.c
do 文字数を数える
in int型 1つ char型配列1つ
out int型 1つ
author 16024003
day 2016.11.30
other none
*****/
#include <stdio.h>
int main(void)
{
    char str[11];
    int i;

    printf("10文字以内の文字列を入力してください--->");
    scanf("%s",str);

    while(str[i] !='\0')
    {
        i++ ;
    }

    printf("入力された文字列の長さは%d文字です\n",i);

    return 0;
}
\end{verbatim}
\end{itembox}

\subsection{実行結果}
\begin{itembox}[l]{実行結果}
\begin{verbatim}
$ ./e111_2
10文字以内の文字列を入力してください--->test
入力された文字列の長さは4文字です
\end{verbatim}
\end{itembox}

\subsection{考察}
文字列の最後の見えないnull文字を調べれることがわかった。


\section{基礎課題2 Lecture5演習1-5}
\subsection{ソースコード}
\begin{itembox}[l]{ソースコード}
\begin{verbatim}
/*****
name e111_4
do 入力された文字を逆順に出力
in int型 1つ char型配列1つ
out  char型配列1つ
author 16024003
day 2016.11.30
other none
*****/
#include <stdio.h>
#include <string.h>
int main(void)
{
    char str[11];
    int i;

    printf("10文字以内の文字を入力してください\n--->");
    scanf("%s",str);
    printf("逆順に出力すると\n");

    for(i = strlen(str)-1;i >= 0; i--)
    {
        printf("%c",str[i]);
    }

    printf("となります\n");

    return 0;

}
\end{verbatim}
\end{itembox}

\subsection{実行結果}
\begin{itembox}[l]{実行結果}
\begin{verbatim}
$ ./e111_4
10文字以内の文字を入力してください
--->test
逆順に出力すると
tsetとなります
\end{verbatim}
\end{itembox}

\subsection{考察}
for文を使って終端からアクセスしていくことで逆順に出力できることがわかった
\section{感想}
配列に文字を入れることで様々な操作が出きることがわかった。
\end{document}























