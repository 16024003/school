%==========================================================================
%Template File
%   Copyright (C) 2006-2009                                
%           by Shinya Watanabe(sin@csse.muroran-it.ac.jp) 
%==========================================================================
\documentclass[a4j,titlepage]{jarticle}
\usepackage{programing_report}
\usepackage{itembkbx}
\begin{document}

%--------------------
%以下に実験レポートのタイトル,自分のクラス名,学籍番号,氏名,提出日を書く.
%--------------------

%%\title{レポートタイトル}を記述する.
\title{第9回 プログラミング演習 レポート}

%%\author{クラス名}{学籍番号}{氏名}を記述する.
\author{前半クラス16024003}{赤堀 冴太朗}

%%\date{提出する年月日}を記述する.
\date{2016 年 12 月15 日}
\maketitle

%--------------------
%以下から本文を開始する.
%--------------------

\section{基礎課題1 Lecture6練習問題1}
\subsection{ソースコード}
\begin{breakitembox}[l]{ソースコード}
\begin{verbatim}

/*****
name renshu4-1
do x,yの計算
in int型配列 2つ
out x,y
author 16024003
day 2016.12.7
other none
*****/
#include <stdio.h>
int Delta(int a,int b,int c,int d);

int main(void)
{
    int f1[3],f2[3],i;
    double x,y;

    printf("2つの方程式を入力してください\nax+by=h ");
    for(i=0;i<3;i++)
    {
        scanf("%d",&f1[i]);
    }
    printf("cx+dy=k ");
        for(i=0;i<3;i++)
    {
        scanf("%d",&f2[i]);
    }

    if(Delta(f1[0],f1[1],f2[0],f2[1])==0)
    {
        printf("ERROR!ad-bc=0になる連立方程式です\n");
    }
    else 
    {
        x=(f1[2]*f2[1]-f1[1]*f2[2])/Delta(f1[0],f1[1],f2[0],f2[1]);
        y=(f1[0]*f2[2]-f1[2]*f2[0])/Delta(f1[0],f1[1],f2[0],f2[1]);
        printf("xの値は%f,yの値は%fです。\n",x,y);
    }
    return 0;
}

int Delta(int a,int b,int c,int d)
{
    int delta;

    delta=a*d-b*c;
    return delta;
}


\end{verbatim}
\end{breakitembox}

\subsection{実行結果}
\begin{itembox}[l]{実行結果}
\begin{verbatim}
$ ./renshu4-1
2つの方程式を入力してください
ax+by=h 1 2 3
cx+dy=k 1 2 3
0ERROR!ad-bc=0になる連立方程式です

$ ./renshu4-1
2つの方程式を入力してください
ax+by=h 1 2 3
cx+dy=k 4 5 6
-3xの値は-1.000000,yの値は2.000000です。
\end{verbatim}
\end{itembox}

\subsection{考察}
関数に要素を4つ与えるのが手間だった。

\section{基礎課題2 Lecture6練習問題2}
\subsection{ソースコード}
\begin{breakitembox}[l]{ソースコード}
\begin{verbatim}

/*****
name renshu4-2.c
do x,yの計算
in int型配列 2つ
out x,y
author 16024003
day 2016.11.9
other none 
*****/
#include <stdio.h>
int product(int n,int m);

int Delta(int g1[2] ,int g2[2]);

int main(void)
{
    int f1[3],f2[3],i,delta;
    double x,y;

    i=0;

    printf("2つの方程式を入力してください\nax+by=h ");
    for(i=0;i<3;i++)
    {
        scanf("%d",&f1[i]);
    }
    printf("cx+dy=k ");
        for(i=0;i<3;i++)
    {
        scanf("%d",&f2[i]);
    }

    delta=Delta(f1,f2);

    if(delta==0)
    {
        printf("ERROR!ad-bc=0になる連立方程式です\n");
    }
    else 
    {
        x=(f1[2]*f2[1]-f1[1]*f2[2])/delta;
        y=(f1[0]*f2[2]-f1[2]*f2[0])/delta;
        printf("xの値は%f,yの値は%fです。\n",x,y);
    }
    return 0;
}

int Delta(int g1[2] ,int g2[2])
{
    int delta;

    delta=g1[0]*g2[1]-g1[1]*g2[0];
    return delta;
}
\end{verbatim}
\end{breakitembox}

\subsection{実行結果}
\begin{itembox}[l]{実行結果}
\begin{verbatim}
$ ./renshu4-2
2つの方程式を入力してください
ax+by=h 1 2 3
cx+dy=k 1 2 3
ERROR!ad-bc=0になる連立方程式です

$ ./renshu4-2
2つの方程式を入力してください
ax+by=h 1 2 3
cx+dy=k 4 5 6
xの値は-1.000000,yの値は2.000000です。
\end{verbatim}
\end{itembox}

\subsection{考察}
ポインタを使おうとしたがうまくいかなかった。ポインタの使い方がよく理解できない。
\section{感想}
関数に配列を渡すときの使いかたが分かった。
\end{document}























