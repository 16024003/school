%==========================================================================
%Template File
%   Copyright (C) 2006-2009                                
%           by Shinya Watanabe(sin@csse.muroran-it.ac.jp) 
%==========================================================================
\documentclass[a4j,titlepage]{jarticle}
\usepackage{programing_report}
\usepackage{itembkbx}
\begin{document}

%--------------------
%以下に実験レポートのタイトル,自分のクラス名,学籍番号,氏名,提出日を書く.
%--------------------

%%\title{レポートタイトル}を記述する.
\title{第11回 プログラミング演習 レポート}

%%\author{クラス名}{学籍番号}{氏名}を記述する.
\author{前半クラス16024003}{赤堀 冴太朗}

%%\date{提出する年月日}を記述する.
\date{2017 年 1 月 12 日}
\maketitle

%--------------------
%以下から本文を開始する.
%--------------------

\section{基礎課題1 lectuer7 演習3ー2}
\subsection{ソースコード}
\begin{breakitembox}[l]{ソースコード}
\begin{verbatim}

/*****
name e73-2.c
do ing型数値の入れ替え
in none
out int型2つ
author 16024003
day 2016.12.7
other none
*****/
#include<stdio.h>
#include<stdlib.h>      /*mallocの利用*/
void swapi(int *x,int *y);

int main(void)
{
    int *a,*b;
    a = (int *)malloc(sizeof(int)*1);   /*メモリ領域の確保*/
    b = (int *)malloc(sizeof(int)*1);
    *a=5,*b=10;                         /*aとbへの代入*/
    printf("x=%d y=%d\n",*a,*b);
    swapi(a,b);

    return 0;

}

void swapi(int *x,int *y)
{
    int temp;
    temp=*x;                            /*tempへ一時的に格納*/
    *x=*y;
    *y=temp;

    printf("x=%d y=%d\n",*x,*y);        /*そのままprintfを返す*/
}
\end{verbatim}
\end{breakitembox}

\subsection{実行結果}
\begin{itembox}[l]{実行結果}
\begin{verbatim}
$ ./e73-2
x=5 y=10
x=10 y=5
\end{verbatim}
\end{itembox}

\subsection{考察}
int *x=5という書き方では代入できなかった。

\section{基礎課題2 lectuer7 演習3-3}
\subsection{ソースコード}
\begin{breakitembox}[l]{ソースコード}
\begin{verbatim}

/*****
name e73-3.c
do  char型文字の入れ替え
in none
out char型2つ
author 16024003
day 2016.11.9
other none 
*****/
#include<stdio.h>
#include<stdlib.h>
void swaps(char *a,char *b);

int main(void)
{
    char *a,*b;    
    a = (char *)malloc(sizeof(char)*1);   /*メモリ領域の確保*/
    b = (char *)malloc(sizeof(char)*1);
    *a='A',*b='B';                        /*aとbへの代入*/
    printf("a=%c b=%c\n",*a,*b);
    swaps(a,b);

    return 0;
}

void swaps(char *a,char *b)
{
    char temp;
    temp=*a;                               /*tempへ一時的に格納*/
    *a=*b;
    *b=temp;

    printf("a=%c b=%c\n",*a,*b);            /*そのままprintfを返す*/
}
\end{verbatim}
\end{breakitembox}

\subsection{実行結果}
\begin{itembox}[l]{実行結果}
\begin{verbatim}
$ ./e73-3
a=A b=B
a=B b=A
\end{verbatim}
\end{itembox}

\subsection{考察}
メモリを確保しなくてはコアダンプしてしまった。

\section{応用課題 lectuer7 演習2-4}
\subsection{ソースコード}
\begin{breakitembox}[l]{ソースコード}
\begin{verbatim}

/*****
name e7-4.c
do バブルソート
in none
out int型配列2つ
author 16024003
day 2016.12.7
other none
*****/
#include<stdio.h>

int main(void)
{
    int i,tmp,flag=1;
    int a[9] ={2,1,7,3,6,4,8,5,9};
    int *p=a;
    while(flag == 1)                    /*flag=1のあいだ繰り返す*/
    {
        flag=0;
        for(i=0;i<8;i++)
        {
            if(*(p+i)<*(p+(i+1)))
            {
                tmp = *(p+i);
                *(p+i) = *(p+(i+1));
                *(p+(i+1)) = tmp;
                flag=1;                 /*ソートがあった場合falg=1*/
            }
        }
    }
    for(i=0;i<=8;i++)
    {
        printf("%d",*(p+i));
    }
    printf("\n");

    return 0;
}
\end{verbatim}
\end{breakitembox}

\subsection{実行結果}
\begin{itembox}[l]{実行結果}
\begin{verbatim}
$ ./e72-4
987654321
\end{verbatim}
\end{itembox}

\subsection{考察}
初期値を指定しなくても動くときとしてしないと動かないときがあった。

\section{感想}
やはりポインタよく分からなかった。

\end{document}























