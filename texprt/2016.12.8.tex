%==========================================================================
%Template File
%   Copyright (C) 2006-2009                                
%           by Shinya Watanabe(sin@csse.muroran-it.ac.jp) 
%==========================================================================
\documentclass[a4j,titlepage]{jarticle}
\usepackage{programing_report}
\usepackage{itembkbx}
\begin{document}

%--------------------
%以下に実験レポートのタイトル,自分のクラス名,学籍番号,氏名,提出日を書く.
%--------------------

%%\title{レポートタイトル}を記述する.
\title{第8回 プログラミング演習 レポート}

%%\author{クラス名}{学籍番号}{氏名}を記述する.
\author{前半クラス16024003}{赤堀 冴太朗}

%%\date{提出する年月日}を記述する.
\date{2016 年 12 月8 日}
\maketitle

%--------------------
%以下から本文を開始する.
%--------------------

\section{基礎課題1 Lecture6演習1-3}
\subsection{ソースコード}
\begin{breakitembox}[l]{ソースコード}
\begin{verbatim}

/*****
name e61_2.c
do 連立方程式の計算
in int型配列2つ
out x,yの解
author 16024003
day 2016.12.7
other none
*****/
#include <stdio.h>
int product(int n,int m);

int main(void)
{
    int f1[3],f2[3],i,delta;
    double x,y;

    printf("2つの方程式を入力してください\nax+by=h ");
    for(i=0;i<3;i++)
    {
        scanf("%d",&f1[i]);
    }
    printf("cx+dy=k ");
        for(i=0;i<3;i++)
    {
        scanf("%d",&f2[i]);
    }

    delta=product(f1[0],f2[1])-product(f1[1],f2[0]);

    if(delta==0)
    {
        printf("ERRAR!ad-bc=0になる連立方程式です\n");
    }
    else 
    {
        x=(product(f1[2],f2[1])-product(f1[1],f2[2]))/delta;
        y=(product(f1[0],f2[2])-product(f1[2],f2[0]))/delta;
        printf("xの値は%f,yの値は%fです。\n",x,y);
    }
    return 0;
}

演習
int product(int n,int m)
{
    int h;

    h=n*m;
    return h;
}
\end{verbatim}
\end{breakitembox}

\subsection{実行結果}
\begin{itembox}[l]{実行結果}
\begin{verbatim}
$ ./e61_2
2つの方程式を入力してください
ax+by=h 2 4 7
cx+dy=k 4 8 9
ERRAR!ad-bc=0になる連立方程式です

$ ./e61_2
2つの方程式を入力してください
ax+by=h 1 2 3
cx+dy=k 4 5 6
xの値は-1.000000,yの値は2.000000です。
\end{verbatim}
\end{itembox}

\subsection{考察}
解を求める式の規則性がわからなかった

\section{基礎課題2 Lecture6演習2-3}
\subsection{ソースコード}
\begin{breakitembox}[l]{ソースコード}
\begin{verbatim}

/*****
name e63
do 任意の数字のソート
in int型配列1つ,int型1つ
out 小さい数字からのソート
author 16024003
day 2016.12.7
other none 
*****/
#include <stdio.h>
void swap(int *x, int *y);

main (void){
    int n,i,j,flag=1;
    printf("入力する数字の個数を入力してください");
    scanf("%d",&n);
    int m[n];
    printf("数字を入力してください");
    for(i=0;i<n;i++){
        scanf("%d",&m[i]);
    }

    while(flag>0){
        flag=0;
        for(i=0;i<n-1;i++){
            if(m[i]>m[i+1]){
                swap(&m[i],&m[i+1]);
                flag++;
            }
        }
    }
    printf("小さい順に並べると ");
    for(i=0;i<n;i++){
        printf("%d ",m[i]);
    }
    printf("です\n");

    return 0;
}

void swap(int *x, int *y){
    int temp;
    
    temp=*x;
    *x=*y;
    *y=temp;

}
\end{verbatim}
\end{breakitembox}

\subsection{実行結果}
\begin{itembox}[l]{実行結果}
\begin{verbatim}
$ ./e63
入力する数字の個数を入力してください5
数字を入力してください3
5
43
7
23
小さい順に並べると 3 5 7 23 43 です
\end{verbatim}
\end{itembox}

\subsection{考察}
ポインタの使い方がいまいち理解できなかった
\section{感想}
比較的コードが長くなってしまって、デバックが大変だった。
\end{document}























